\documentclass{article}

% Standard Packages
\usepackage[utf8]{inputenc} % Specify input encoding
\usepackage{amsmath}        % For advanced math typesetting (e.g., align environment)
\usepackage{amssymb}        % For math symbols
\usepackage{amsfonts}       % For math fonts
\usepackage{geometry}       % For page layout customization
\geometry{a4paper, margin=1in} % Set page size to A4 and margins to 1 inch

% Custom commands from the document
% \sector command is used for denoting economic sectors like T, H, I.
% \textnormal ensures upright text, suitable for use in math mode.
\newcommand{\sector}[1]{\textnormal{#1}}

\begin{document}


\section{Model: A Three-Sector Framework with Evolving Automation}
\label{sec:model}
We develop a discrete-time model with three sectors to analyze the economic transition towards near-Artificial General Intelligence (AGI). 
The economy consists of a Traditional sector ($\sector{T}$), a Human-centric sector ($\sector{H}$), and an Intelligence sector ($\sector{I}$). 
The key dynamic element is the evolving substitutability between AI capital and human labor, particularly in the Traditional and Intelligence sectors, driven by technological progress. 
Time is indexed by $t=0, 1, 2, \dots$.

\subsection{Economic Environment and Factors of Production}
The economy is populated by a representative household that supplies a fixed total amount of labor $L$ inelastically. 
There are three types of inputs: traditional physical capital ($K$), AI-specific capital ($A$), and labor ($L$). 
Both types of capital are accumulated through investment and depreciate over time. 
Labor can be allocated across the three production sectors or remain unemployed ($U$). 
We denote the allocation of labor at time $t$ as $L_{j,t}$ for sector $j \in \{\sector{T}, \sector{H}, \sector{I}\}$ and $L_{U,t}$ for unemployment, such that $L_{\sector{T},t} + L_{\sector{H},t} + L_{\sector{I},t} + L_{U,t} = L$. 
Similarly, capital stocks are sector-specific: $K_{j,t}$ and $A_{j,t}$ (where applicable).

\subsection{Production Technologies}
Each sector $j \in \{\sector{T}, \sector{H}, \sector{I}\}$ produces a homogeneous final good using sector-specific production technologies. 
We assume perfect competition in all markets, implying factors are paid their marginal products.
Total output in the economy is $Y_t = Y_{\sector{T},t} + Y_{\sector{H},t} + Y_{\sector{I},t}$.

\subsubsection{Traditional Sector (T)}
Output in the Traditional sector, $Y_{\sector{T},t}$, is produced using traditional capital $K_{\sector{T},t}$, AI capital $A_{\sector{T},t}$, and labor $L_{\sector{T},t}$ via a nested Constant Elasticity of Substitution (CES) function. 
The outer nest combines traditional capital with an aggregate of AI capital and labor, $H_{\sector{T},t}$. 
The inner nest combines AI capital and labor to form this aggregate.
\begin{align}
Y_{\sector{T},t} &= \left[ \alpha_{\sector{T}} (K_{\sector{T},t})^{\hat{\rho}_{\sector{T}}} + (1-\alpha_{\sector{T}}) (H_{\sector{T},t})^{\hat{\rho}_{\sector{T}}} \right]^{1/\hat{\rho}_{\sector{T}}} \label{eq:prod_T_outer} \\
H_{\sector{T},t} &= \left[ \phi_{\sector{T},t} (A_{\sector{T},t})^{\rho_{\sector{T}}} + (1-\phi_{\sector{T},t}) (L_{\sector{T},t})^{\rho_{\sector{T}}} \right]^{1/\rho_{\sector{T}}} \label{eq:prod_T_inner}
\end{align}
Here, $\alpha_{\sector{T}} \in (0,1)$ is the share parameter for traditional capital in the outer nest. 
The parameter $\phi_{\sector{T},t} \in [0,1]$ is the time-varying share parameter for AI capital in the inner nest, capturing the evolving technological frontier and potential for automation. 
$\hat{\rho}_{\sector{T}} \leq 1$ and $\rho_{\sector{T}} \leq 1$ govern the elasticity of substitution in the outer and inner nests, respectively. 
The elasticity of substitution between $K_{\sector{T}}$ and $H_{\sector{T}}$ is $\hat{\sigma}_{\sector{T}} = 1/(1-\hat{\rho}_{\sector{T}})$, and between $A_{\sector{T}}$ and $L_{\sector{T}}$ is $\sigma_{\sector{T}} = 1/(1-\rho_{\sector{T}})$. 
A higher $\sigma_{\sector{T}}$ implies greater substitutability between AI and labor within this sector.

\subsubsection{Human Sector (H)}
The Human sector produces output $Y_{\sector{H},t}$ using traditional capital $K_{\sector{H},t}$ and labor $L_{\sector{H},t}$. 
Tasks in this sector are assumed to be inherently difficult or undesirable to automate, reflecting human-specific capabilities. 
We model this with a standard CES production function that does not include AI capital as a direct input:
\begin{equation}
Y_{\sector{H},t} = \left[ \alpha_{\sector{H}} (K_{\sector{H},t})^{\rho_{\sector{H}}} + (1-\alpha_{\sector{H}}) (L_{\sector{H},t})^{\rho_{\sector{H}}} \right]^{1/\rho_{\sector{H}}} \label{eq:prod_H}
\end{equation}
where $\alpha_{\sector{H}} \in (0,1)$ is the capital share parameter, and $\rho_{\sector{H}} \leq 1$ determines the elasticity of substitution $\sigma_{\sector{H}} = 1/(1-\rho_{\sector{H}})$ between capital and labor in this sector.

\subsubsection{Intelligence Sector (I)}
Output in the Intelligence sector, $Y_{\sector{I},t}$, is produced using traditional capital $K_{\sector{I},t}$ (representing infrastructure like servers and buildings), AI capital $A_{\sector{I},t}$ (representing models, software, compute), and labor $L_{\sector{I},t}$ (representing tasks like programming, data analysis). 
Similar to the Traditional sector, we use a nested CES structure:
\begin{align}
Y_{\sector{I},t} &= \left[ \alpha_{\sector{I}} (K_{\sector{I},t})^{\hat{\rho}_{\sector{I}}} + (1-\alpha_{\sector{I}}) (H_{\sector{I},t})^{\hat{\rho}_{\sector{I}}} \right]^{1/\hat{\rho}_{\sector{I}}} \label{eq:prod_I_outer} \\
H_{\sector{I},t} &= \left[ \phi_{\sector{I},t} (A_{\sector{I},t})^{\rho_{\sector{I}}} + (1-\phi_{\sector{I},t}) (L_{\sector{I},t})^{\rho_{\sector{I}}} \right]^{1/\rho_{\sector{I}}} \label{eq:prod_I_inner}
\end{align}
The parameters $\alpha_{\sector{I}}$, $\phi_{\sector{I},t}$, $\hat{\rho}_{\sector{I}}$, and $\rho_{\sector{I}}$ are defined analogously to those in the Traditional sector, with $\phi_{\sector{I},t}$ also evolving over time. 
The corresponding elasticities are $\hat{\sigma}_{\sector{I}} = 1/(1-\hat{\rho}_{\sector{I}})$ and $\sigma_{\sector{I}} = 1/(1-\rho_{\sector{I}})$. 
We will assume that labor and AI is substitutes in the Intelligence sector, so $\sigma_{\sector{I}} > \sigma_{\sector{T}}$.

\subsection{Factor Payments}
Factors of production are mobile across sectors (subject to frictions for labor) and are paid their marginal products. 
Let $w_{j,t}$, $r_{K,j,t}$, and $r_{A,j,t}$ denote the wage rate, the rental rate of traditional capital, and the rental rate of AI capital in sector $j$ at time $t$, respectively.
\begin{align}
w_{j,t} &= \frac{\partial Y_{j,t}}{\partial L_{j,t}} \quad \text{for } j \in \{\sector{T}, \sector{H}, \sector{I}\} \label{eq:wage} \\
r_{K,j,t} &= \frac{\partial Y_{j,t}}{\partial K_{j,t}} \quad \text{for } j \in \{\sector{T}, \sector{H}, \sector{I}\} \label{eq:rental_K} \\
r_{A,j,t} &= \frac{\partial Y_{j,t}}{\partial A_{j,t}} \quad \text{for } j \in \{\sector{T}, \sector{I}\} \quad (\text{Note: } r_{A,\sector{H},t} = 0 \text{ by assumption}) \label{eq:rental_A}
\end{align}
The marginal products are derived from the production functions (\ref{eq:prod_T_outer})-(\ref{eq:prod_I_inner}). 
For the nested structures ($\sector{T}$ and $\sector{I}$), the chain rule applies. 
For example, the marginal product of labor in sector $\sector{T}$ is:
\begin{equation}
\frac{\partial Y_{\sector{T},t}}{\partial L_{\sector{T},t}} = \frac{\partial Y_{\sector{T},t}}{\partial H_{\sector{T},t}} \cdot \frac{\partial H_{\sector{T},t}}{\partial L_{\sector{T},t}}
\end{equation}
where
\begin{align}
\frac{\partial Y_{\sector{T},t}}{\partial H_{\sector{T},t}} &= (1-\alpha_{\sector{T}}) \left( \frac{Y_{\sector{T},t}}{H_{\sector{T},t}} \right)^{1-\hat{\rho}_{\sector{T}}} \\
\frac{\partial H_{\sector{T},t}}{\partial L_{\sector{T},t}} &= (1-\phi_{\sector{T},t}) \left( \frac{H_{\sector{T},t}}{L_{\sector{T},t}} \right)^{1-\rho_{\sector{T}}}
\end{align}
Similar expressions hold for $\partial Y_{j,t}/\partial K_{j,t}$ and $\partial Y_{j,t}/\partial A_{j,t}$ using the appropriate derivatives of the CES functions.\footnote{We assume standard CES derivative forms, handling edge cases (e.g., $\rho \to 0$ for Cobb-Douglas, $\rho \to 1$ for linear) as implemented in the corresponding simulation code.}

\subsection{Capital Accumulation and Investment Allocation}
Aggregate output in the economy is $Y_t = Y_{\sector{T},t} + Y_{\sector{H},t} + Y_{\sector{I},t}$. 
A constant fraction $s_K$ of aggregate output is saved and invested in traditional capital, and a fraction $s_A$ is invested in AI capital. 
Total investment funds are $Inv_{K,t} = s_K Y_t$ and $Inv_{A,t} = s_A Y_t$.

These aggregate investment funds are allocated across sectors based on the relative marginal productivities of capital. 
Investment flows to sectors offering higher returns, moderated by a sensitivity parameter $\eta \geq 0$.
\begin{align}
I_{K,j,t} &= Inv_{K,t} \cdot \frac{ \max(0, r_{K,j,t})^\eta }{ \sum_{k \in \{\sector{T,H,I}\}} \max(0, r_{K,k,t})^\eta } \quad \text{for } j \in \{\sector{T}, \sector{H}, \sector{I}\} \label{eq:inv_alloc_K} \\
I_{A,j,t} &= Inv_{A,t} \cdot \frac{ \max(0, r_{A,j,t})^\eta }{ \sum_{k \in \{\sector{T,I}\}} \max(0, r_{A,k,t})^\eta } \quad \text{for } j \in \{\sector{T}, \sector{I}\} \label{eq:inv_alloc_A}
\end{align}
If the denominator is zero (e.g., all relevant marginal products are non-positive), investment allocation is zero or determined by an alternative rule (e.g., equal shares). We set $I_{A,\sector{H},t} = 0$.

Capital stocks evolve according to the standard law of motion, with depreciation rates $\delta_K$ and $\delta_A$:
\begin{align}
K_{j,t+1} &= (1 - \delta_K) K_{j,t} + I_{K,j,t} \quad \text{for } j \in \{\sector{T}, \sector{H}, \sector{I}\} \label{eq:k_accum} \\
A_{j,t+1} &= (1 - \delta_A) A_{j,t} + I_{A,j,t} \quad \text{for } j \in \{\sector{T}, \sector{I}\} \label{eq:a_accum}
\end{align}

\subsection{Labor Market Dynamics}
Labor allocation evolves based on wage differentials and labor market frictions. 
The total labor force $L$ is constant. 
The dynamics involve flows between employment in sectors $\sector{T}$, $\sector{H}$, $\sector{I}$, and unemployment $\sector{U}$.

\noindent
\textbf{Job Separation:} A fraction $\lambda_s \in [0,1]$ of workers employed in each sector $j$ exogenously separates into unemployment: $S_{j,t} = \lambda_s L_{j,t}$.

\noindent
\textbf{Hiring from Unemployment:} Unemployed workers find jobs based on available vacancies, implicitly driven by sector wages. 
A fraction $\lambda_f \in [0,1]$ of the unemployment pool $L_{U,t}$ finds jobs in period $t$. 
The total number of hires from unemployment is $H_{U,t} = \lambda_f L_{U,t}$. 
These hires are allocated across sectors $j \in \{\sector{T,H,I}\}$ proportionally to their relative wage attractiveness, governed by a sensitivity parameter $\xi \geq 0$:
\begin{equation}
    H_{j,t} = H_{U,t} \cdot \frac{ \max(0, w_{j,t})^\xi }{ \sum_{k \in \{\sector{T,H,I}\}} \max(0, w_{k,t})^\xi } \label{eq:labor_hire}
\end{equation}
If the denominator is zero, hires are allocated according to an alternative rule.

\noindent
\textbf{Inter-Sectoral Mobility:} Employed workers may switch sectors seeking higher wages. 
We assume a fraction $\chi \in [0,1]$ of workers are immobile. 
The remaining fraction $(1-\chi)$ can potentially move. 
The net flow between sectors depends on wage differentials $(w_{k,t} - w_{j,t})$ and is scaled by a mobility factor $\mu \in [0,1]$ reflecting adjustment costs or search frictions. 
Sectors with higher relative wages attract workers from sectors with lower relative wages, subject to the availability of mobile workers in the source sector and the overall friction captured by $\mu$. 
Let $M_{k \to j, t}$ denote the flow of mobile workers from sector $k$ to sector $j$. 
This flow is increasing in $w_{j,t}$ relative to $w_{k,t}$ and constrained by $(1-\chi)L_{k,t}$ and $\mu$. 
The exact functional form follows the implementation in the simulation code, capturing limited arbitrage across sectors.

The labor stock in each sector evolves according to:
\begin{equation}
L_{j,t+1} = L_{j,t} - S_{j,t} + H_{j,t} + \sum_{k \neq j} (M_{k \to j, t} - M_{j \to k, t}) \quad \text{for } j \in \{\sector{T}, \sector{H}, \sector{I}\} \label{eq:labor_accum_j}
\end{equation}
The unemployment stock evolves as:
\begin{equation}
L_{U,t+1} = L_{U,t} + \sum_{j \in \{\sector{T,H,I}\}} S_{j,t} - H_{U,t} \label{eq:labor_accum_U}
\end{equation}
The dynamics ensure labor conservation: $\sum_{j \in \{\sector{T,H,I}\}} L_{j,t+1} + L_{U,t+1} = L$.

\subsection{Evolution of Technology: The Automation Parameter \texorpdfstring{$\phi_t$}{phi\_t}}
A key driver of the model dynamics is the exogenous evolution of the AI share parameters, $\phi_{\sector{T},t}$ and $\phi_{\sector{I},t}$, which capture the increasing capability and cost-effectiveness of AI technology in substituting for or complementing labor in the Traditional and Intelligence sectors. 
We assume these parameters follow a logistic (S-curve) path over time, representing an initial phase of slow adoption, followed by rapid advancement, and eventually saturation near a maximum potential level. 
For $j \in \{\sector{T}, \sector{I}\}$:
\begin{equation}
\phi_{j,t} = \phi_{j,\text{min}} + \frac{\phi_{j,\text{max}} - \phi_{j,\text{min}}}{1 + e^{-\gamma_j (t - t_{0,j})}} \label{eq:phi_evolution}
\end{equation}
where $\phi_{j,\text{min}}$ is the initial share, $\phi_{j,\text{max}} \in (\phi_{j,\text{min}}, 1]$ is the maximum potential AI share, $\gamma_j > 0$ controls the speed of transition, and $t_{0,j}$ is the midpoint of the transition phase for sector $j$. 
This exogenous path reflects the anticipated trajectory towards near-AGI. The parameter $\phi_{\sector{H},t}$ for the Human sector is implicitly fixed at 0.

\subsection{Model Summary}
The model describes the evolution of capital stocks ($K_j, A_j$) and labor allocations ($L_j, L_U$) across three sectors, driven by investment decisions based on marginal returns and labor flows influenced by wage differentials and frictions. 
The core dynamic is the changing nature of production, particularly in the $\sector{T}$ and $\sector{I}$ sectors, due to the S-curve evolution of the AI technology parameter $\phi_t$. 
The interplay between technology adoption ($\phi_t$), capital-labor/AI-labor substitutability ($\rho$'s), investment ($s_K, s_A, \eta$), and labor market frictions ($\lambda_s, \lambda_f, \chi, \mu, \xi$) determines the path of output, wages, and employment structure as the economy approaches a state of near-AGI.

\end{document}